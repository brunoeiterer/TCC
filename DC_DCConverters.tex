\chapter{Funcionamento e Modelagem dos Conversores CC-CC} \label{secao:conversores_cc_cc}

Para possibilitar o controle do ponto de operação do painel são empregados conversores CC-CC com diferentes técnicas de controle. As mais comuns são o casamento de impedância entre a carga e a impedância vista pelos terminais do painel \cite{haroun2015}, \cite{ramasamy2014} e o controle direto da tensão nos terminais do painel, quando a carga é um barramento com tensão constante \cite{xiao2007topology}, \cite{xiao2007}.

\section{Casamento de Impedância}

A impedância vista pelos terminais do painel determina o ponto de operação, de acordo com a equação \ref{eq:impedancia_painel}. Por conta deste fator, dificilmente um painel opera no \gls{mpp} quando conectado a uma carga arbitrária em um ambiente arbitrário.

\begin{equation}
R = \frac{V}{I}
\label{eq:impedancia_painel}
\end{equation}

Para exemplificar o funcionamento dos conversores CC-CC no controle da impedância vista pelos terminais do painel solar vamos considerar o caso de ser necessária tensão mais baixa na carga do que no painel, no qual é utilizado um conversor \textit{Buck}. O circuito deste sistema pode ser visto na figura \ref{circuito_com_conversor_buck}.

\begin{figure}[!htpb]
\begin{center}
\begin{circuitikz} [american]
\draw
(0,0) to[pvsource, l = V\textsubscript{pv}] (0,3)
      to[nos, l = S] (2,3)
(0,0) -- (2,0) to[diode, l = D]
(2,3) to[inductor, l = L] (4,3)
(4,3) to[capacitor, l = C] (4,0) -- (2,0)
(4,3) to[short] (6,3)
(6,3) to[resistor, l=R\textsubscript{L}] (6,0) -- (4,0)
(6,3) -- (8,3) to[open, v^=V\textsubscript{L}, o-o] (8,0) -- (6,0);
\draw[->] (-0.25, 2) -- (-0.25, 2.75) node[midway, left] {I\textsubscript{pv}};
\end{circuitikz}
\end{center}
\caption{Circuito com conversor \textit{Buck} para casamento de impedância}
\label{circuito_com_conversor_buck}
\end{figure}

As equações \ref{eq:ganho_estatico_conversor_buck} e \ref{eq:ganho_estatico_conversor_buck_corrente} representam o ganho estático de tensão e corrente do conversor \textit{Buck}.

\begin{equation} \label{eq:ganho_estatico_conversor_buck}
\frac{V_{L}}{V_{pv}} = D
\end{equation}

\begin{equation}\label{eq:ganho_estatico_conversor_buck_corrente}
\frac{I_{pv}}{I_{L}} = D
\end{equation}

Combinando estas duas equações na equação \ref{eq:relacao_impedancia_entrada_saida_buck} obtemos a relação entre a impedância de entrada e saída do conversor. Como a impedância de entrada é a impedância de saída do painel (carga do painel), podemos variar a razão cíclica D e consequentemente variar o ponto de operação do painel, de forma a sempre operar no ponto de máxima potência.

\begin{equation} \label{eq:relacao_impedancia_entrada_saida_buck}
\begin{aligned}
R &= \frac{V_{pv}}{I_{pv}} \\
&= \frac{V_{L}}{D} \cdot \frac{1}{I_{L}\cdot D} \\
&= \frac{1}{D^{2}} \cdot \frac{V_{L}}{I_{L}} \\
&= \frac{1}{D^{2}} \cdot R_{L}
\end{aligned}
\end{equation}

\section{Controle Direto da Tensão}

Para exemplificar o controle da tensão do painel solar e vamos considerar o caso em que a tensão na saída é maior do que na entrada, no qual é utilizado um conversor \textit{Boost}. O circuito completo pode ser visto na figura \ref{circuito_completo}.

\begin{figure}[!htpb]
\begin{center}
\begin{circuitikz} [american]
\draw
(0,0) to[pvsource, l = V\textsubscript{pv}] (0,3)
	  to[short](2,3)
(2,3) to[capacitor, l = C\textsubscript{in}](2,0) -- (0,0)
(2,3) to[inductor, l = L] (4,3)
(4,3) to[nos, l = S] (4,0) -- (2,0)
(4,3) to[diode, l = D] (6,3)
(6,3) to[capacitor, l = C\textsubscript{out}] (6,0) -- (4,0)
(6,3) to[short] (8,3)
(8,3) to[battery, v = V\textsubscript{bat}] (8,0) -- (6,0);

\draw[->] (-0.25, 2) -- (-0.25, 2.75) node[midway, left] {I\textsubscript{pv}};
\end{circuitikz}
\end{center}
\caption{Circuito com conversor \textit{Boost} para controle da tensão}
\label{circuito_completo}
\end{figure}

Considerando todos os componentes ideais pode-se fazer a análise do circuito com o objetivo de se obter a relação entre a tensão da bateria e a tensão do painel a partir do fato de que a tensão média no indutor é nula.

Na primeira etapa, com a chave fechada, temos o circuito da figura \ref{circuito_primeira_etapa}. A tensão no indutor nesta etapa é dada na equação \ref{tensao_indutor_primeira_etapa}

\begin{figure}[!htpb]
\begin{center}
\begin{circuitikz} [american]
\draw
(0,0) to[pvsource, l = V\textsubscript{pv}] (0,3)
(0,3) to[short] (2,3)
(2,3) to[capacitor, l = C\textsubscript{in}] (2,0) -- (0,0)
(2,3) to[inductor, l = L] (4,3)
(4,3) to[short] (4,0) -- (2,0)
(4,3) to[open] (6,3)
(6,3) to[capacitor, l = C\textsubscript{out}] (6,0) -- (4,0)
(6,3) to[short] (8,3)
(8,3) to[battery, v = V\textsubscript{bat}] (8,0) -- (6,0);

\draw[->] (-0.25, 2) -- (-0.25, 2.75) node[midway, left] {I\textsubscript{pv}};
\end{circuitikz}
\end{center}
\caption{Circuito da Primeira Etapa}
\label{circuito_primeira_etapa}
\end{figure}

\begin{equation} \label{tensao_indutor_primeira_etapa}
V_{L} = V_{pv}
\end{equation}

Na segunda etapa a chave é aberta, com isto a corrente no indutor é reduzida, causando uma inversão na polaridade da tensão no indutor, de forma que o diodo é polarizado e passa a conduzir. O circuito desta etapa está na figura \ref{circuito_segunda_etapa}. A tensão no indutor é dada pela equação \ref{tensao_indutor_segunda_etapa}.

\begin{figure}[!htpb]
\begin{center}
\begin{circuitikz} [american]
\draw
(0,0) to[pvsource, l = V\textsubscript{pv}] (0,3)
(0,3) to[short] (2,3)
(2,3) to[capacitor, l = C\textsubscript{in}] (2,0) -- (0,0)
(2,3) to[inductor, l = L] (4,3)
(4,3) to[capacitor, l = C \textsubscript{out}] (4,0) -- (2,0)
(4,3) to[short] (6,3)
(6,3) to[battery, v = V\textsubscript{bat}] (6,0) -- (4,0);

\draw[->] (-0.25, 2) -- (-0.25, 2.75) node[midway, left] {I\textsubscript{pv}};
\end{circuitikz}
\end{center}
\caption{Circuito da Segunda Etapa}
\label{circuito_segunda_etapa}
\end{figure}

\begin{equation} \label{tensao_indutor_segunda_etapa}
V_{L} = V_{pv} - V_{bat}
\end{equation}

Considerando que a chave ficar fechada pelo tempo DT e aberta pelo tempo (1-D)T, onde D é a razão cíclica e T o período a tensão média no indutor é dada pela equação \ref{tensao_media_indutor}.

\begin{equation} \label{tensao_media_indutor}
\begin{aligned}
V_{L} &= \frac{1}{T}\left(\int_{0}^{DT} V_{pv}\,dt + \int_{DT}^{T} V_{pv} - V_{bat}\,dt\right) \\
&= \frac{1}{T}\left[V_{pv}\cdot DT + (V_{pv}-V_{bat})(T-DT)\right] \\
&= V_{pv}\cdot D + V_{pv} - V_{pv}\cdot D - V_{bat} +V_{bat}\cdot D \\
&= V_{pv} + V_{bat}(D-1)
\end{aligned}
\end{equation}

Como a tensão média no indutor é nula temos:

\begin{equation}
\begin{gathered}
V_{pv} + V_{bat}(D-1) = 0 \\
V_{pv} = V_{bat}(1-D) 
\end{gathered}
\end{equation}

Com isso vemos que é possível controlar a tensão do painel diretamente pelo controle da razão cíclica do conversor.

Esta estratégia de controle direto com conversor \textit{Boost} será a implementada neste trabalho, devido ao fato de que os painéis trabalham em torno de \SI{5}{\volt} e as duas baterias em série apresentam uma tensão de \SI{5.4}{\volt} quando estão descarregadas ao limite inferior.