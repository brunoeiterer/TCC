\chapter{Introdução}

A primeira ideia da criação de satélites artificiais surgiu em 1728, de Isaac Newton, no terceiro volume da obra \textit{Philosophi\ae Naturalis Principia Mathematica} (Os Princípios Matemáticos da Filosofia Natural), chamado de \textit{De Mundi Systemate} (Sobre o Sistema do Mundo), no qual Newton propôs que um tiro de canhão poderia entrar em órbita da terra caso fosse disparado de uma montanha bastante elevada a uma velocidade específica, chamada de velocidade orbital \cite{DeMundiSystemate}.

Em 1903, Konstantin Eduardovich Tsiolkovsky publicou o trabalho Exploração Espacial Usando Propulsão a Jato, no qual apareceram os cálculos da velocidade orbital e como um foguete multiestágio poderia atingir esta velocidade \cite{rocketEquation}. Outro trabalho teórico que surgiu em seguida é O problema da Viagem Espacial (Das Problem der Befahrung des Weltraums - der Raketen-Motor, em alemão), escrito por Herman Poto\v{c}nik, que trazia as ideias de utilizar satélites para observação terrestre, experimentos ciêntificos, satélites geoestacionários, comunicações através de rádios e até uma ideia preliminar de uma estação espacial \cite{theProblemOfSpaceTravel}. Em 1945, Arthur C. Clarke publicou o artigo Transmissores extra-terrestres - Estações em Foguetes podem Proporcionar Cobertura de Rádio Mundial? (\textit{Extra-Terrestrial Relays – Can Rocket Stations Give Worldwide Radio Coverage?}), no qual foi apresentada a ideia de se utilizar satélites geoestacionários para comunicação \cite{extraTerrestrialRelays}.

Em 1957 foi lançado pela União Soviética o primeiro satélite artificial, chamado de Sputnik 1, o qual foi utilizado para medir a densidade das camadas superiores da atmosfera através do empuxo aplicado sobre ele. O seu sucesso levou os Estados Unidos da América a investirem pesado no setor aeroespacial, dando início a chamada corrida espacial \cite{SputnikReconsidered}. Com isso, em 1961 já existiam mais de 100 satélites em órbita da terra \cite{orbitalDebrisChronology}.

Atualmente existem mais de 1000 satélites operacionais em órbita. Eles são utilizados em várias aplicações muito comuns na vida moderna, como comunicação (internet, celulares, transmissões de TV), observação da terra, defesa e sistemas de posicionamento global (GPS, do inglês \textit{global positioning system}) \cite{satelliteHitsAtlantic}.

Apesar disso, apenas alguns países e empresas conseguem desenvolver e lançar satélites, devido a sua complexidade e principalmente o elevadíssimo custo. Por conta disso, estão sendo desenvolvidos novos padrões para satélites, de forma a propiciar a oportunidade para pequenas empresas e até universidades de participarem do desenvolvimento espacial \cite{largeBenefitsOfSmallSatellites}. Dentre estes novos padrões foi criado em 1999, por Jordi Puig-Suari e Bob Twiggs o \textit{Cubesat} \cite{cubesatCreation}, que é um satélite em forma de cubo com arestas de \SI{10}{\centi\metre} e massa menor ou igual a \SI{1.33}{\kilo\gram} \cite{cubesatDesignSpecification}.

\textit{Cubesats} apresentam novos desafios de desenvolvimento, como orçamentos limitados para seu desenvolvimento, possível criação de muito lixo espacial, caso muitos \textit{cubesats} venham a ser lançados, e principalmente limitação na área disponível para captação e armazenamento de energia \cite{nanosatelliteDevelopment}.

Tendo em vista esta limitação no aspecto energético, neste trabalho será apresentado o módulo de energia do \textit{cubesat} Floripa-Sat, atualmente em desenvolvimento na UFSC, com foco no gerenciamento da energia, desde a entrada de energia até o consumo dos outros módulos.

\section{OBJETIVOS}

Esta seção apresenta o objetivo geral e os objetivos específicos deste trabalho.

\subsection{Geral}

Modelar, simular e testar o funcionamento do módulo de energia do nanossatélite Floripa-Sat.

\subsection{Específicos}
\begin{itemize}
\item Modelar os painéis solares
\item Modelar o conversor boost da entrada
\item Modelar o consumo do \textit{cubesat}
\item Simular o sistema completo
\item Realizar testes com o sistema real
\item Comparar os resultados reais com os simulados
\end{itemize}

\section{Organização do trabalho}