\chapter{Modelagem das Cargas}

\section{Computador de Bordo}

O computador de bordo tem funções como armazenamento dos dados em memória não-volátil, leitura de sensores e aquisição dos dados dos demais módulos do satélite.

Alguns contribuintes para o consumo estão listados na tabela \ref{consumo_computador_bordo}. Como é possível observar, poucos componentes contribuem para a maior parte do consumo. O cálculo da potência consumida é realizado multiplicando a corrente consumida pela tensão de alimentação, de \SI{3.3}{\volt}, exceto em casos especiais onde a equação é fornecida pelo \textit{datasheet} do componente.

\begin{table}[!htpb]
\centering
\begin{tabular}{c c c c}
\\ \hline
Componente & Quantia & Corrente [\SI{}{\milli\ampere}] & Potência [\SI{}{\milli\watt}] \\ \hline \hline
\glsentryshort{imu} (MPU-9250) & 1 & 3.7 \cite{mpu9250} & 12.21 \\
\glsentryshort{imu} (BMX055) & 1 & 5.7 \cite{bmx055} & 18.81 \\
Gerador de Referência & 1 & 0.026 \cite{ref5030}, \cite{msp430f6659} & 0.0008 \cite{ref5030} \\
Amplificador Operacional & 4 & 0.2 \cite{tlv341} & 2.64 \\
\textit{Watchdog} Externo & 1 & 0.025 \cite{tps3823} & 0.0825 \\
microSD & 1 & 0.25 \cite{microSD} & 0.825 \\
Memória não-volátil & 3 & 0.05 \cite{is25lp128} & 0.495 \\
Microcontrolador & 1 & 8.39 \cite{msp430f6659} & 57.1134 \cite{msp430f6659} \\
Sensor de Corrente & 1 & 0.23 \cite{max9934} & 2.277 \\
Resistor Shunt (\SI{0.05}{\ohm}) & 1 & 19.271 & 0.01857 \\ \hline
Total & - & 19.271 & 94.47 \\ \hline
\end{tabular}
\caption{Consumo do Computador de Bordo}
\label{consumo_computador_bordo}
\end{table}

Aqui foram considerados apenas os consumos constantes, ou seja, o aumento no consumo durante comunicações do processador com outros dispositivos não foi considerado por apresentar um valor muito pequeno quando comparado com o constante (\SI{0.05}{\milli\watt} de média).

\section{Rádios e Amplificadores de Potência}

Os dois rádios do satélite têm como função enviar os dados coletados para a terra, através de telemetria. Durante a transmissão o consumo de potência é muito maior do em outros momentos, devido ao fato de que o \gls{pa} de cada rádio fica desativado. Portanto para modelar o consumo deste sistema será considerado apenas o comportamento dinâmico, de acordo com a tabela \ref{consumo_radios}. Como os dados enviados por cada rádio são diferentes, os períodos de transmissão e o tempo ativo também são diferentes.

\begin{table}[!htpb]
\centering
\begin{tabular}{c c c c}
\\ \hline
Componente & Período/ & Corrente [\SI{}{\ampere}] & Potência [\SI{}{\milli\watt}] \\
& Tempo Ativo [\SI{}{\second}] & & \\ \hline \hline
\glsentryshort{pa} (Transceiver) & 60/2 & 0.6 \cite{rf6886} & 1.98 \\
\glsentryshort{pa} (Beacon) & 10/0.6 & 0.6 \cite{rf6886} & 1.98 \\ \hline
Total & - & - & - \\ \hline
\end{tabular}
\caption{Consumo dos Rádios}
\label{consumo_radios}
\end{table}