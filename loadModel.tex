\chapter{Modelagem das Cargas} \label{secao:modelagem_cargas}

\section{Computador de Bordo}

O computador de bordo tem funções como armazenamento dos dados em memória não-volátil, leitura de sensores e aquisição dos dados dos demais módulos do satélite.

Alguns contribuintes para o consumo estão listados na tabela \ref{consumo_computador_bordo}. Como é possível observar, poucos componentes contribuem para a maior parte do consumo. O cálculo da potência consumida é realizado multiplicando a corrente consumida pela tensão de alimentação, de \SI{3.3}{\volt}, exceto em casos especiais onde a equação é fornecida pelo \textit{datasheet} do componente.

\begin{table}[!htpb]
\centering
\begin{tabular}{c c c c}
\\ \hline
Componente & Quantia & Corrente [\SI{}{\milli\ampere}] & Potência [\SI{}{\milli\watt}] \\ \hline \hline
\glsentryshort{imu} (MPU-9250) & 1 & 3.7 \cite{mpu9250} & 12.21 \\
\glsentryshort{imu} (BMX055) & 1 & 5.7 \cite{bmx055} & 18.81 \\
Gerador de Referência & 1 & 0.026 \cite{ref5030}, \cite{msp430f6659} & 0.0008 \cite{ref5030} \\
Amplificador Operacional & 4 & 0.2 \cite{tlv341} & 2.64 \\
\textit{Watchdog} Externo & 1 & 0.025 \cite{tps3823} & 0.0825 \\
microSD & 1 & 0.25 \cite{microSD} & 0.825 \\
Memória não-volátil & 3 & 0.05 \cite{is25lp128} & 0.495 \\
Microcontrolador & 1 & 8.39 \cite{msp430f6659} & 57.1134 \cite{msp430f6659} \\
Sensor de Corrente & 1 & 0.23 \cite{max9934} & 2.277 \\
Resistor Shunt (\SI{0.05}{\ohm}) & 1 & 19.271 & 0.01857 \\ \hline
Total & - & 19.271 & 94.47 \\ \hline
\end{tabular}
\caption{Consumo do Computador de Bordo}
\label{consumo_computador_bordo}
\end{table}

Aqui foram considerados apenas os consumos constantes, ou seja, o aumento no consumo durante comunicações do processador com outros dispositivos não foi considerado por apresentar um valor muito pequeno quando comparado com o constante (\SI{0.05}{\milli\watt} de média).

Além disso, este módulo pode ser desativo dependendo do nível de carga armazenado nas baterias, de acordo com a seção \ref{secao:variacao_niveis_energia}.

\section{Rádios e Amplificadores de Potência}

Os dois rádios do satélite têm como função enviar os dados coletados para a terra, através de telemetria. Durante a transmissão o consumo de potência é muito maior do que em outros momentos, devido ao fato de que o \gls{pa} de cada rádio fica desativado. Portanto para modelar o consumo deste sistema será considerado apenas o comportamento dinâmico, de acordo com a tabela \ref{consumo_radios}. Cada rádio envia periodicamente dados para a Terra, com períodos de transmissão e o tempo ativo diferentes, sendo que o período de transmissão é variável de acordo com o nível de carga armazenado nas baterias (seção \ref{secao:variacao_niveis_energia}). Além disso existe a possibilidade de a estação terrestre solicitar o envio de dados através de telecomandos. Neste caso, a transmissão fica ativa até que todos os dados solicitados sejam enviados. Para modelar este comportamento será considerada uma janela de 10 minutos por órbita, o que simula um hipotético pior caso (em termos de energia consumida).

\begin{table}[!htpb]
\centering
\begin{tabular}{c c c c}
\\ \hline
Componente & Período/ & Corrente [\SI{}{\ampere}] & Potência [\SI{}{watt}] \\
& Tempo Ativo [\SI{}{\second}] & & \\ \hline \hline
\glsentryshort{pa} (Transceiver) & 60-\(\infty\)/2 & 0.6 \cite{rf6886} & 1.98 \\
\glsentryshort{pa} (Beacon) & 10-30/0.6 & 0.6 \cite{rf6886} & 1.98 \\ \hline
Total & - & - & - \\ \hline
\end{tabular}
\caption{Consumo dos Rádios}
\label{consumo_radios}
\end{table}

\section{Sistema de Energia}

O \gls{eps} tem como função captar a energia dos painéis solares, armazenar em baterias e distribuir para os outros módulos. Estas funções geram o consumo detalhado na tabela \ref{consumo_eps}.

\begin{table}[!htpb]
\centering
\begin{tabular}{c c c c}
\\ \hline
Componente & Quantia & Corrente [\SI{}{\milli\ampere}] & Potência [\SI{}{\milli\watt}] \\ \hline \hline
Gerador de Referência & 1 & 0.026 \cite{ref5030}, \cite{msp430f6659} & 0.0008 \cite{ref5030} \\
Amplificador Operacional & 5 & 0.2 \cite{tlv341} & 0.66 \\
Sensor de Corrente & 7 & 0.23 \cite{max9934} & 0.759 \\
Resistor Shunt (\SI{0.75}{\ohm}) & 1 & 0.2 & 3 \\
Resistor Shunt (\SI{0.05}{\ohm}) & 6 & 508 & 12.903 \\
Timer 555 & 1 & 0.250 \cite{lmc555} & 0.825 \\
ADC externo & 1 & 0.59 \cite{ads1248} & 1.947 \\
Microcontrolador & 1 & 8.39 \cite{msp430f6659} & 57.1134 \cite{msp430f6659} \\
Kill-Switches & 4 & 350 & 3.063 \cite{si4403} \\
Monitor de Bateria & 1 & 0.135 \cite{ds2775} & 1.134 \cite{ds2775} \\
Proteção das Baterias & 1 & 700 & 10.29 \cite{fds6898az} \\
Aquecedor das Baterias & 2 & - & 3180\\
Conversor CC-CC (5420) & 1 & 49.456 & 206.588 \cite{tps5420} \\
Conversor CC-CC (5410) & 1 & 21.281 & 210.322 \cite{tps5410} \\
Conversor CC-CC (54540) & 2 & 600 & 149.032 \cite{tps54540} \\ \hline
Total & - & - & - \\ \hline
\end{tabular}
\caption{Consumo do Sistema de Energia}
\label{consumo_eps}
\end{table}

O Conversor CC-CC 5420 tem como função alimentar a parte digital dos rádios e o Sistema de Energia, portanto seu consumo é constante. Da mesma forma, o Conversor CC-CC 5410 tem como função alimentar o Computador de Bordo, portanto seu consumo também é constante. Por fim, o Conversor CC-CC 54540 tem como função alimentar o \gls{pa} de cada rádio, de forma que o seu consumo é dinâmico, com comportamento igual ao dos rádios. O aquecedor das baterias só é necessário nos momentos da órbita onde há eclipse, portanto assim será considerado para a modelagem.

\section{\textit{Payloads}}

Os \textit{Payloads} são os módulos do satélite que realizam as funções principais. Por exemplo, no caso de um satélite que tira fotos da terra os \textit{payloads} são as câmeras. No caso deste trabalho o \textit{payload} é um \gls{fpga} usado para testar os efeitos da radição nos componentes no espaço, com consumo de \SI{288}{\milli\ampere} em \SI{5}{\volt} (\SI{1.44}{\watt}), devido a presença do \gls{fpga} Xilinx Artix 7 XC7A200T.


\section{Variação com os Níveis de Energia Armazenada} \label{secao:variacao_niveis_energia}

Para evitar que as baterias sejam descarregadas muito rapidamente e o satélite fique inoperante por muito tempo foram determinados níveis de carga baseados no nível de carga armazenado nas baterias. Estes níveis podem ser vistos na tabela \ref{niveis_carga}.

\begin{table}[!htpb]
\centering
\begin{tabular}{c c c c c}
\\ \hline
Nível & Período \textit{Beacon} [\SI{}{\second}] & Período \textit{Downlink} & \textit{Payloads} & OBDH \\ 
& &  Periódico [\SI{}{\second}] & & \\ \hline \hline
L1 & 10 & 60 & on & on \\
L2 & 10 & 60 & off & on \\
L3 & 20 & 120 & off & on \\
L4 & 30 & \(\infty\) & off & on \\
L5 & 30 & \(\infty\) & off & off \\ \hline
\end{tabular}
\caption{Níveis de Carga do Sistema}
\label{niveis_carga}
\end{table}

A relação entre os limiares de carga armazenada e a transição entre os níveis da tabela \ref{niveis_carga} estão na tabela \ref{niveis_energia}. Os limiares de transição de nível alto-baixo e baixo-alto são diferentes para evitar oscilações do sistema em torno destes limiares.

\begin{table}[!htpb]
\centering
\begin{tabular}{c c}
\\ \hline
Transição & Carga Armazenada [\SI{}{\ampere\hour}] \\ \hline \hline
L1-L2 & 2.4 \\
L2-L1 & 2.55 \\
L2-L3 & 1.8 \\
L3-L2 & 1.95 \\
L3-L4 & 1.2 \\
L4-L3 & 1.35 \\
L4-L5 & 0.6 \\
L5-L4 & 0.75 \\ \hline
\end{tabular}
\caption{Transição dos Níveis de Carga do Sistema}
\label{niveis_energia}
\end{table}