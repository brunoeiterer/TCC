\chapter{Conclusões} \label{secao:conclusoes}

Em relação a modelagem e simulação realizada, podemos considerar que os painéis estão próximos o suficiente do real, como demonstrado primeiramente no capítulo \ref{secao:painel solar} e depois nos capítulos \ref{secao:simulacao_sistema} e \ref{secao:teste_sistema}. Uma possível melhoria futura seria separar cada painel na simulação, de forma que cada um teria seu controle \gls{mppt} próprio, aproximando assim a simulação da realidade. Isto ainda não foi realizado pelo motivo de que cada \gls{mppt} introduz um \textit{Loop} Algébrico (\textit{Algebraic Loop}) na simulação, e sua solução requer métodos mais robustos e melhor controle da ferramenta.

O modelo das cargas foi utilizado tanto na simulação quanto no teste real. Esta decisão foi tomada baseado no fato de que a simulação ainda não havia sido validada. Portanto não seria possível verificar quais são as falhas na simulação caso o sistema completo fosse utilizado. A validação deste modelo será realizada em trabalhos futuros.

Outro ponto interessante é utilizar um modelo físico do conversor \textit{Boost} no lugar da fonte de tensão controlada para facilitar a simulação de não-idealidades. Apesar de que o modelo representou bem o controle da tensão do painel na simulação, as perdas introduzidas pelo conversor no teste real não apareceram na simulação, o que gerou uma discrepância nos resultados. Esta melhoria também será realizada em trabalhos futuros.

De modo geral, pode-se considerar que os objetivos do trabalho foram alcançados, sendo que o sistema foi validado e agora existem um modelo e uma simulação que podem ser utilizados para se ter uma idéia geral do funcionamento do sistema, assim como podem ser utilizados para projetos de novos sistemas no futuro.