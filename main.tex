\documentclass[a5paper, oldfontcommands, hidelinks]{ufsc-thesis}  % escolha o tamanho do papel aqui

\usepackage{cmap}
\usepackage[utf8]{inputenc}
\usepackage[T1]{fontenc}

\usepackage{textcomp}

%---- Bibliography packages ----%

\usepackage[style=numeric, sorting=none]{biblatex}
\DeclareNameAlias{author}{last-first}
\addbibresource{bibliography.bib}

%---- Units Packages ----%

\usepackage{siunitx}
\sisetup{output-decimal-marker = {,}}

%---- Drawing Packages ----%

\usepackage{tikz}
\usepackage{circuitikz}
\usepackage{pgfplots}
\pgfplotsset{compat=1.5}

\usetikzlibrary{positioning}

\usepackage{tikz}
\usetikzlibrary{shapes,arrows}

\tikzstyle{decision} = [diamond, draw, fill=black!20, 
    text width=4.5em, text badly centered, node distance=3cm, inner sep=0pt]
\tikzstyle{process} = [rectangle, draw, fill=black!20, 
    text width=5em, text centered, rounded corners, minimum height=4em]
\tikzstyle{line} = [draw, -latex']


%---- Lists Packages ----%
\usepackage[acronym, automake, nomain, nonumberlist]{glossaries-extra}

% Preâmbulo
\titulo{Modelagem, Simulação e Testes de um Sistema de Energia Aplicado a Nanossatélites}
\autor{Bruno Vale Barbosa Eiterer}
\data{2017}
\instituicao{Universidade Federal de Santa Catarina}
\local{Florianópolis}
\tipotrabalho{}
\orientador{Prof. Eduardo Augusto Bezerra, Ph.D.}
\coorientador{Leonardo Kessler Slongo, Ms.C.}
\programa{Engenharia Elétrica}
\preambulo{}
\centro{}
\assuntos{}

\setabbreviationstyle[acronym]{long-short}
\makeglossaries
\loadglsentries[\acronymtype]{glossary}

\addto\captionsbrazil{\renewcommand\listfigurename{Lista de Figuras}}
\addto\captionsbrazil{\renewcommand\listtablename{Lista de Tabelas}}

\begin{document}
% Inicia parte pré-textual do documento capa, folha de rosto, folha de
% aprovação, aprovação, resumo, lista de tabelas, lista de figuras, etc.
\pretextual%
\imprimircapa%!
%\imprimirfolhaderosto%
\begin{center}
Bruno Vale Barbosa Eiterer \\
\vspace{\fill}
\Large{Modelagem, Simulação e Testes de um Sistema de Energia Aplicado a Nanossatélites} \\
\vspace{\fill}
\end{center}
\hfill
\begin{minipage}{0.5\textwidth}
\raggedright \small Trabalho de Conclusão de Curso submetido à Universidade Federal de Santa Catarina, como parte dos requisitos necessários para a obtenção do Grau de Bacharel em Engenharia Elétrica. \\
Orientador: Prof. Eduardo Augusto Bezerra, Ph.D. \\
Coorientador: Leonardo Kessler Slongo, Ms.C.
\end{minipage}
\vfill
\begin{center}
Florianópolis \\
2017
\end{center}


\clearpage
\imprimirfichacatalografica%
%\tableofcontents%
\textual%

\clearpage
\begin{center}
\large\textbf{AGRADECIMENTOS}
\end{center}
À equipe do EPS do FloripaSat, Leonardo Slongo, Sara Vega e Túlio Gomes, por todos as discussões, testes e por todos os momentos passados juntos nesses quatro anos. Aos demais membros do FloripaSat, por todo o companheirismo. Aos meu pais, Berenice e Antonio Carlos, por todo seu apoio. Ao meu irmão, Caio, por toda a ajuda durante o curso. À minha irmã, Letícia, por todas as risadas, séries e músicas compartilhadas. Ao professor e orientador Eduardo Bezerra, pela orientação em todos os anos de trabalho. 

\clearpage
\begin{center}
\large\textbf{RESUMO}
\end{center}
Atualmente satélites fazem parte do cotidiano das pessoas, ainda que de forma invisível. Apesar disso, o custo de desenvolvimento torna inacessível o projeto realizado por grande parte das empresas e universidades. Por isso, foi criado o padrão \textit{Cubesat}, que introduziu o conceito de satélites em forma de cubo de \SI{10}{\centi\metre} de aresta. Este padrão apresenta novos desafios, principalmente na captação de energia, devido a pequena área disponível para painéis solares. Além disso, apesar do custo reduzido, os baixos orçamentos também são um desafio. Portanto é necessário realizar modelagens e simulações antes de se iniciar o projeto do satélite em si, de forma a reduzir os custos. Este trabalho busca modelar, simular e validar o sistema de energia de um nanossatélite, para facilitar projetos futuros.

\noindent\textbf{Palavras-Chave:} Nanossatélites. Sistemas de Energia. Modelagem. Simulação.

\clearpage
\begin{center}
\large\textbf{ABSTRACT}
\end{center}
Nowadays satellites are a daily part of people life, even if they are invisible. In spite of that, the development cost makes projects inaccessible for the majority of companies and universities. Because of that it was created the Cubesat standard, which introduced the concept of \SI{10}{\centi\metre} edge cube satellites. This standard has new challenges, specially on energy harvesting, due to the small area available for solar panels. Besides that, in spite of the reduced cost, the tight budgets are also a challenge.  Because of that it is necessary to model and perform simulations before starting the real project, to reduce the costs. This work  presents modelling, simulation and validation of a nanossatelite energy system, to make future projects easier.

\noindent\textbf{Keywords:} Nanosatellites. Energy Systems. Modelling. Simulation.

\clearpage
\listoffigures
\clearpage
\listoftables
\glsaddall 
\printglossary[type=\acronymtype, title = Lista de Abreviaturas e Siglas]
\clearpage
\setcounter{tocdepth}{3} % + subsubsections
\tableofcontents

\chapter{Introdução}

A primeira ideia da criação de satélites artificiais surgiu em 1728, de Isaac Newton, no terceiro volume da obra \textit{Philosophi\ae Naturalis Principia Mathematica} (Os Princípios Matemáticos da Filosofia Natural), chamado de \textit{De Mundi Systemate} (Sobre o Sistema do Mundo), no qual Newton propôs que um tiro de canhão poderia entrar em órbita da terra caso fosse disparado de uma montanha bastante elevada a uma velocidade específica, chamada de velocidade orbital \cite{newton1728}.

Em 1903, Konstantin Eduardovich Tsiolkovsky publicou o trabalho Exploração Espacial Usando Propulsão a Jato, no qual apareceram os cálculos da velocidade orbital e como um foguete multiestágio poderia atingir esta velocidade \cite{maul2012}. Outro trabalho teórico que surgiu em seguida é O problema da Viagem Espacial (Das Problem der Befahrung des Weltraums - der Raketen-Motor, em alemão), escrito por Herman Poto\v{c}nik, que trazia as ideias de utilizar satélites para observação terrestre, experimentos ciêntificos, satélites geoestacionários, comunicações através de rádios e até uma ideia preliminar de uma estação espacial \cite{potocnik1929}. Em 1945, Arthur C. Clarke publicou o artigo Transmissores extra-terrestres - Estações em Foguetes podem Proporcionar Cobertura de Rádio Mundial? (\textit{Extra-Terrestrial Relays – Can Rocket Stations Give Worldwide Radio Coverage?}), no qual foi apresentada a ideia de se utilizar satélites geoestacionários para comunicação \cite{Clarke1945}.

Em 1957 foi lançado pela União Soviética o primeiro satélite artificial, chamado de Sputnik 1, o qual foi utilizado para medir a densidade das camadas superiores da atmosfera através do empuxo aplicado sobre ele. O seu sucesso levou os Estados Unidos da América a investirem pesado no setor aeroespacial, dando início a chamada corrida espacial \cite{McQuaid2017}. Com isso, em 1961 já existiam mais de 100 satélites em órbita da terra \cite{Portree1999}.

Atualmente existem mais de 4000 satélites operacionais em órbita. Eles são utilizados em várias aplicações muito comuns na vida moderna, como comunicação (internet, celulares, transmissões de TV), observação da terra, defesa e \gls{gps}\cite{spaceObjectsIndex2017}.

Apesar disso, apenas alguns países e empresas conseguem desenvolver e lançar satélites, devido a sua complexidade e principalmente o elevadíssimo custo. Por conta disso, estão sendo desenvolvidos novos padrões para satélites, de forma a propiciar a oportunidade para pequenas empresas e até universidades de participarem do desenvolvimento espacial \cite{Baker2008}. Dentre estes novos padrões foi criado em 1999, por Jordi Puig-Suari e Bob Twiggs o \textit{Cubesat} \cite{Messier2015}, que é um satélite em forma de cubo com arestas de \SI{10}{\centi\metre} e massa menor ou igual a \SI{1.33}{\kilo\gram} \cite{cubesatDesignSpecification2014}.

Um dos novos desafios de desenvolvimento para \textit{Cubesats} é a limitação no orçamento, a qual torna necessária a realização de modelagens e simulações antes do desenvolvimento do produto, de forma a reduzir os custos. Outro possível desafio é a criação de muito lixo espacial, caso muitos \textit{cubesats} venham a ser lançados. Por fim, existe uma limitação na área disponível para captação e armazenamento de energia devido ao tamanho dos satélites\cite{Kalman2011}. Tendo em vista esta limitação no aspecto energético, é necessário projetar o sistema cuidadosamente de forma que a energia fornecida pelos painéis solares e armazenadas nas baterias seja suficiente para alimentar as cargas do sistema.

Neste trabalho será apresentado o módulo de energia do \textit{cubesat} Floripa-Sat, atualmente em desenvolvimento na UFSC, com foco no gerenciamento da energia, desde a entrada de energia até o consumo dos outros módulos. Este módulo de energia tem como entrada de energia paineis solares, que são conectados através de um conversor \textit{boost} à duas baterias Li-Ion conectadas em série. Um algortimo \gls{mppt} é utilizado para operar os paineis com máxima eficiência. O sistema distribui a energia para os outros módulos do satélite (computador de bordo, sistema de comunicação e \textit{payloads}) em diferentes tensões através de conversores CC-CC integrados. Será realizada a modelagem do sistema, simulações e por fim testes com o sistema real, de forma que será possível validar a simulação para trabalhos futuros assim como verificar se o sistema projetado está de acordo com o necessário.

\section{Objetios}

Esta seção apresenta o objetivo geral e os objetivos específicos deste trabalho.

\subsection{Geral}

Modelar, simular e testar o funcionamento do módulo de energia do nanossatélite Floripa-Sat.

\subsection{Específicos}
\begin{itemize}
\item Modelar os painéis solares
\item Modelar o conversor boost da entrada
\item Modelar o consumo do \textit{cubesat}
\item Simular o sistema completo
\item Realizar testes com o sistema real
\item Comparar os resultados reais com os simulados
\end{itemize}

\section{Organização do trabalho}

A seção \ref{secao:painel solar} apresenta brevemente o funcionamento dos paineis solares, um circuito equivalente com componentes representando a conversão de energia solar em energia elétrica e as perdas, um modelo do painel real a ser utilizado no trabalho e por fim algoritmos para operação do painel com máxima eficiência.

Na seção \ref{secao:conversores_cc_cc} são apresentadas duas maneiras de se utilizar conversores CC-CC no controle da operação de paineis solares e uma modelagem simplificada a ser utilizada na simulação do sistema.

O próximo passo é a modelagem das cargas do sistema (seção \ref{secao:modelagem_cargas}), para possibilitar a verificação do funcionamento do sistema na simulação. Para tal foram utilizadas informações fornecidas nos \textit{datasheets} dos principais componentes de cada módulo do satélite (comunicação, computador de bordo, sistema de energia e \textit{payloads}).

Nas seções \ref{secao:simulacao_sistema} e \ref{secao:teste_sistema} são apresentados as simulações e testes realizados na bancada para verificar se o sistema projetado atende os requisitos de carga e também validar a simulação realizada em comparação com o sistema real.

Ao final do trabalho são apresentadas as conclusões sobre o que foi proposto e desenvolvido (seção \ref{secao:conclusoes}).




\chapter{Painel Solar}

O funcionamento de células solares é baseado no efeito fotovoltaico, ou seja, a geração de tensão ou corrente elétrica a partir da incidência de luz. O efeito fotovoltaico ocorre da seguinte maneira: em um semicondutor ideal existem dois níveis de energia que elétrons podem ser excitados, representados pela camada de valência, com energia baixa, e a camada de condução, com energia mais alta. Entre estas duas camada existe o chamado \textit{bandgap}, uma região de com níveis de energia que os elétrons não podem ter. Quando um fóton entra em contato com o semicondutor os elétrons na camada de valência absorvem sua energia e passam para a camada de condução, gerando uma corrente elétrica. Como não é possível que os elétrons possuam níveis intermediários de energia fótons que não possuam energia superior ao \textit{bandgap} não são absorvidos e passam sem interagir com a célula.\cite{jager2014}.

\section{Circuito Equivalente de um Painel Solar}

Em uma célula real existem perdas causadas por aquecimento no movimento dos elétrons, por impurezas no material que geram novos níveis de energia dentro do \textit{bandgap} e também por recombinação na junção p-n \cite{blakers2013}. Portanto para representar as células solares através de um circuito equivalente é necessário uma fonte de corrente associada com alguns componentes que representam as perdas.

O modelo para uma célula aqui utilizado é apresentado em \cite{erdem2013}. O circuito equivalente composto por uma fonte de corrente I\textsubscript{ph}, um diodo D, uma resistência paralela R\textsubscript{p} e uma resistência série R\textsubscript{s}. Este circuito pode ser encontrado na figura \ref{modelo_celula_solar}, onde I\textsubscript{ph} é a corrente fotogerada do painel, I\textsubscript{D} é a corrente do diodo, I\textsubscript{R\textsubscript{p}} é a corrente na resistência paralela, I é a corrente da célula e V é a tensão da célula.

\begin{figure}[!htpb]
\begin{center}
\begin{circuitikz} [american]
\draw
(0,0) to[I, l = I\textsubscript{ph}] (0,3) -- (2,3)
      to[diode, l = D] (2,0) -- (0,0)
(2,3) to[short] (4,3)
(4,3) to[resistor, l = R\textsubscript{p}] (4,0) -- (2,0)
(4,3) to [short] (4.5,3)
(4.5,3) to[resistor, l = R\textsubscript{s}] (6.5,3)
	  to[short, -o] (7,3)
(4,0) to[short, -o] (7,0)
(7,3) to[open, v=V] (7,0);
\draw[->] (6.25, 3.25) -- (7,3.25) node[midway, above] {I};
\draw[->] (2.25, 2.75) -- (2.25, 2) node[midway, right] {I\textsubscript{D}};
\draw[->] (4.25, 2.75) -- (4.25, 2) node[midway, right] {I\textsubscript{R\textsubscript{p}}};
\end{circuitikz}
\end{center}
\caption{Circuito equivalente da célula solar}
\label{modelo_celula_solar}
\end{figure}

A partir da análise do circuito equivalente, temos a seguinte relação entre a corrente e a tensão da celula:

\begin{equation} \label{eq:relacao_corrente_tensao_celula}
I = I_{ph} - I_{D} - I_{R_{p}}
\end{equation}

Por simplicidade e sem perda de precisão I\textsubscript{ph} pode ser determinada diretamente pela corrente de curto-circuito I\textsubscript{sc} do painel, respeitando-se a dependência com a irradiância E e a temperatura da célula T\textsubscript{c} (equação \ref{eq:corrente_fotogerada}), assim pode-se obtê-la diretamente dos \textit{datasheets} fornecidos pelos fabricantes. 

\begin{equation} \label{eq:corrente_fotogerada}
I_{ph} = I_{sc}\cdot \frac{E}{E_{0}} \cdot [1 + \Delta_{I_{sc}}(T_{c} - T_{0})]
\end{equation}

A corrente no diodo é dada pela equação \ref{eq:corrente_diodo}, onde I\textsubscript{o} é a corrente de saturação na sombra, q é a carga de um elétron, n é o fator de idealidade, k é a constante de Boltzmann e T\textsubscript{c} é a temperatura da célula \cite{bellia2014}.

\begin{equation} \label{eq:corrente_diodo}
I_{D} = I_{o}(e^{-\frac{q(V+I\cdot R_{s})}{nkT_{c}}}-1)
\end{equation}

A corrente no resistor R\textsubscript{p} pode ser obtida através da equação \ref{eq:corrente_Rp}, conhecida como lei de Ohm.

\begin{equation} \label{eq:corrente_Rp}
I_{R_{p}} = \frac{V+I\cdot R_{s}}{R_{p}}
\end{equation}

Combinando as equações apresentadas, obtem-se a equação \ref{eq:relacao_corrente_tensao_celula_com_parametros}, que mostra a relação entre a corrente e a tensão da celula solar a partir dos parâmetros do circuito equivalente.

\begin{equation} \label{eq:relacao_corrente_tensao_celula_com_parametros}
I = I_{ph} - I_{o}(e^{-\frac{q(V+I\cdot R_{s})}{nkT_{c}}}-1) - \frac{V+I\cdot R_{s}}{R_{p}}
\end{equation}

Quando várias células são conectadas em série e/ou em paralelo é formado um painel solar. As curvas características de corrente por tensão e potência por tensão de um painel podem ser vistas nas figuras \ref{figura_corrente_painel_temperatura} e \ref{figura_potencia_painel_temperatura}, onde é evidenciada a dependência com a temperatura e a irradiância. 

\pgfplotstableread[col sep = comma]{figures/solarPanelCharacteristics-25.csv}\solarPanelCharacteristicsMinusTwentyFive
\pgfplotstableread[col sep = comma]{figures/solarPanelCharacteristics0.csv}\solarPanelCharacteristicsZero
\pgfplotstableread[col sep = comma]{figures/solarPanelCharacteristics25.csv}\solarPanelCharacteristicsTwentyFive
\pgfplotstableread[col sep = comma]{figures/solarPanelCharacteristics50.csv}\solarPanelCharacteristicsFifty

\begin{figure}[!htpb]
\begin{minipage}{.5\textwidth}
\begin{center}
\begin{tikzpicture}[trim axis left, trim axis right]
\begin{axis}[xlabel = {V [\SI{}{\volt}]}, ylabel = {I [\SI{}{\ampere}]}, ymin = 0, yticklabel style={/pgf/number format/fixed}, xtick distance=1, legend pos = south west, scale = 0.5, scale only axis]
\addplot[mark = none, color = cyan] table[x index = {0}, y index = {1}]{\solarPanelCharacteristicsMinusTwentyFive};
\addplot[mark = none, color = blue, dotted] table[x index = {0}, y index = {1}]{\solarPanelCharacteristicsZero};
\addplot[mark = none, color = magenta, dashed] table[x index = {0}, y index = {1}]{\solarPanelCharacteristicsTwentyFive};
\addplot[mark = none, color = red, dash dot] table[x index = {0}, y index = {1}]{\solarPanelCharacteristicsFifty};
\addlegendentry{\SI{-25}{\celsius}}
\addlegendentry{\SI{0}{\celsius}}
\addlegendentry{\SI{25}{\celsius}}
\addlegendentry{\SI{50}{\celsius}}
\end{axis}
\end{tikzpicture}
\caption[caption]{Corrente de um painel \\\hspace{\textwidth} solar}
\label{figura_corrente_painel_temperatura}
\end{center}
\end{minipage}
\begin{minipage}{.5\textwidth}
\begin{center}
\begin{tikzpicture}[trim axis left, trim axis right]
\begin{axis}[xlabel = {V [\SI{}{\volt}]}, ylabel = {I [\SI{}{\watt}]}, ymin = 0, yticklabel style={/pgf/number format/fixed}, xtick distance=1, legend pos = north west, scale = 0.5, scale only axis]
\addplot[mark = none, color = cyan] table[x index = {0}, y index = {2}]{\solarPanelCharacteristicsMinusTwentyFive};
\addplot[mark = none, color = blue, dotted] table[x index = {0}, y index = {2}]{\solarPanelCharacteristicsZero};
\addplot[mark = none, color = magenta, dashed] table[x index = {0}, y index = {2}]{\solarPanelCharacteristicsTwentyFive};
\addplot[mark = none, color = red, dash dot] table[x index = {0}, y index = {2}]{\solarPanelCharacteristicsFifty};
\addlegendentry{\SI{-25}{\celsius}}
\addlegendentry{\SI{0}{\celsius}}
\addlegendentry{\SI{25}{\celsius}}
\addlegendentry{\SI{50}{\celsius}}
\end{axis}
\end{tikzpicture}
\caption[caption]{Potência de um painel \\\hspace{\textwidth} solar}
\label{figura_potencia_painel_temperatura}
\end{center}
\end{minipage}
\end{figure}

\pgfplotstableread[col sep = comma]{figures/solarPanelCharacteristics250.csv}\solarPanelCharacteristicsTwoHundredFifty
\pgfplotstableread[col sep = comma]{figures/solarPanelCharacteristics500.csv}\solarPanelCharacteristicsFiveHundred
\pgfplotstableread[col sep = comma]{figures/solarPanelCharacteristics750.csv}\solarPanelCharacteristicsSevenHundredFifty
\pgfplotstableread[col sep = comma]{figures/solarPanelCharacteristics1000.csv}\solarPanelCharacteristicsThousand

\begin{figure}[!htpb]
\begin{minipage}{.5\textwidth}
\begin{center}
\begin{tikzpicture}[trim axis left, trim axis right]
\begin{axis}[xlabel = {V [\SI{}{\volt}]}, ylabel = {I [\SI{}{\ampere}]}, ymin = 0, yticklabel style={/pgf/number format/fixed}, xtick distance=1, legend pos = south west, scale = 0.5, scale only axis]
\addplot[mark = none, color = cyan] table[x index = {0}, y index = {1}]{\solarPanelCharacteristicsTwoHundredFifty};
\addplot[mark = none, color = blue, dotted] table[x index = {0}, y index = {1}]{\solarPanelCharacteristicsFiveHundred};
\addplot[mark = none, color = magenta, dashed] table[x index = {0}, y index = {1}]{\solarPanelCharacteristicsSevenHundredFifty};
\addplot[mark = none, color = red, dash dot] table[x index = {0}, y index = {1}]{\solarPanelCharacteristicsThousand};
\addlegendentry{\SI[per-mode=symbol]{250}{\watt\per\meter\squared}}
\addlegendentry{\SI[per-mode=symbol]{500}{\watt\per\meter\squared}}
\addlegendentry{\SI[per-mode=symbol]{750}{\watt\per\meter\squared}}
\addlegendentry{\SI[per-mode=symbol]{1000}{\watt\per\meter\squared}}
\end{axis}
\end{tikzpicture}
\caption[caption]{Corrente de um painel \\\hspace{\textwidth} solar}
\label{figura_corrente_painel_irradiância}
\end{center}
\end{minipage}
\begin{minipage}{.5\textwidth}
\begin{center}
\begin{tikzpicture}[trim axis left, trim axis right]
\begin{axis}[xlabel = {V [\SI{}{\volt}]}, ylabel = {I [\SI{}{\watt}]}, ymin = 0, yticklabel style={/pgf/number format/fixed}, xtick distance=1, legend pos = south west, scale = 0.5, scale only axis]
\addplot[mark = none, color = cyan] table[x index = {0}, y index = {2}]{\solarPanelCharacteristicsTwoHundredFifty};
\addplot[mark = none, color = blue, dotted] table[x index = {0}, y index = {2}]{\solarPanelCharacteristicsFiveHundred};
\addplot[mark = none, color = magenta, dashed] table[x index = {0}, y index = {2}]{\solarPanelCharacteristicsSevenHundredFifty};
\addplot[mark = none, color = red, dash dot] table[x index = {0}, y index = {2}]{\solarPanelCharacteristicsThousand};
\addlegendentry{\SI[per-mode=symbol]{250}{\watt\per\meter\squared}}
\addlegendentry{\SI[per-mode=symbol]{500}{\watt\per\meter\squared}}
\addlegendentry{\SI[per-mode=symbol]{750}{\watt\per\meter\squared}}
\addlegendentry{\SI[per-mode=symbol]{1000}{\watt\per\meter\squared}}
\end{axis}
\end{tikzpicture}
\caption[caption]{Potência de um painel \\\hspace{\textwidth} solar}
\label{figura_potência_painel_irradiância}
\end{center}
\end{minipage}
\end{figure}

Como podemos ver pelas figuras, existe um ponto de máxima potência (MPP, do inglês \textit{maximum power point}) e que este ponto varia muito com a variação da temperatura do painel e um pouco com a variação da irradiância incidente, portanto para operar o painel sempre com a maior eficiência é necessário aplicar um controlador conhecido como MPPT (do inglês, \textit{maximum power point tracker}).

\section{MPPT}


\chapter{Funcionamento e Modelagem dos Conversores CC-CC} \label{secao:conversores_cc_cc}

\section{Introdução}

Para possibilitar o controle do ponto de operação do painel são empregados conversores CC-CC com diferentes técnicas de controle. As mais comuns são o casamento de impedância entre a carga e a impedância vista pelos terminais do painel \cite{haroun2015}, \cite{ramasamy2014} e o controle direto da tensão nos terminais do painel, quando a carga é um barramento com tensão constante \cite{xiao2007topology}, \cite{xiao2007}, \cite{coelho2012}. Este capítulo apresenta uma modelagem para cada caso, primeiramente com um conversor \textit{Buck} e depois com um conversor \textit{Boost}.

\section{Casamento de Impedância}

A impedância vista pelos terminais do painel determina o ponto de operação, de acordo com a equação \ref{eq:impedancia_painel}. Por conta deste fator, dificilmente um painel opera no \gls{mpp} quando conectado a uma carga arbitrária em um ambiente arbitrário.

\begin{equation}
R = \frac{V}{I}
\label{eq:impedancia_painel}
\end{equation}

Para exemplificar o funcionamento dos conversores CC-CC no controle da impedância vista pelos terminais do painel solar vamos considerar o caso de ser necessária tensão mais baixa na carga do que no painel, no qual é utilizado um conversor \textit{Buck}. O circuito deste sistema pode ser visto na Figura \ref{circuito_com_conversor_buck}.

\begin{figure}[!htpb]
\begin{center}
\begin{circuitikz} [american]
\draw
(0,0) to[pvsource, l = V\textsubscript{pv}] (0,3)
      to[nos, l = S] (2,3)
(0,0) -- (2,0) to[diode, l = D]
(2,3) to[inductor, l = L] (4,3)
(4,3) to[capacitor, l = C] (4,0) -- (2,0)
(4,3) to[short] (6,3)
(6,3) to[resistor, l=R\textsubscript{L}] (6,0) -- (4,0)
(6,3) -- (8,3) to[open, v^=V\textsubscript{L}, o-o] (8,0) -- (6,0);
\draw[->] (-0.25, 2) -- (-0.25, 2.75) node[midway, left] {I\textsubscript{pv}};
\end{circuitikz}
\end{center}
\caption{Circuito com conversor \textit{Buck} para casamento de impedância}
\label{circuito_com_conversor_buck}
\end{figure}

As equações \ref{eq:ganho_estatico_conversor_buck} e \ref{eq:ganho_estatico_conversor_buck_corrente} representam o ganho estático de tensão e corrente do conversor \textit{Buck}.

\begin{equation} \label{eq:ganho_estatico_conversor_buck}
\frac{V_{L}}{V_{pv}} = D
\end{equation}

\begin{equation}\label{eq:ganho_estatico_conversor_buck_corrente}
\frac{I_{pv}}{I_{L}} = D
\end{equation}

Combinando estas duas equações na equação \ref{eq:relacao_impedancia_entrada_saida_buck} obtemos a relação entre a impedância de entrada e saída do conversor. Como a impedância de entrada é a impedância de saída do painel (carga do painel), podemos variar a razão cíclica D e consequentemente variar o ponto de operação do painel, de forma a sempre operar no ponto de máxima potência.

\begin{equation} \label{eq:relacao_impedancia_entrada_saida_buck}
\begin{aligned}
R &= \frac{V_{pv}}{I_{pv}} \\
&= \frac{V_{L}}{D} \cdot \frac{1}{I_{L}\cdot D} \\
&= \frac{1}{D^{2}} \cdot \frac{V_{L}}{I_{L}} \\
&= \frac{1}{D^{2}} \cdot R_{L}
\end{aligned}
\end{equation}

\section{Controle Direto da Tensão}

Para exemplificar o controle da tensão do painel solar vamos considerar o caso em que a tensão na saída é maior do que na entrada, no qual é utilizado um conversor \textit{Boost}. O circuito completo pode ser visto na Figura \ref{circuito_completo}.

\begin{figure}[!htpb]
\begin{center}
\begin{circuitikz} [american]
\draw
(0,0) to[pvsource, l = V\textsubscript{pv}] (0,3)
	  to[short](2,3)
(2,3) to[capacitor, l = C\textsubscript{in}](2,0) -- (0,0)
(2,3) to[inductor, l = L] (4,3)
(4,3) to[nos, l = S] (4,0) -- (2,0)
(4,3) to[diode, l = D] (6,3)
(6,3) to[capacitor, l = C\textsubscript{out}] (6,0) -- (4,0)
(6,3) to[short] (8,3)
(8,3) to[battery, v = V\textsubscript{bat}] (8,0) -- (6,0);

\draw[->] (-0.25, 2) -- (-0.25, 2.75) node[midway, left] {I\textsubscript{pv}};
\end{circuitikz}
\end{center}
\caption{Circuito com conversor \textit{Boost} para controle da tensão}
\label{circuito_completo}
\end{figure}

Considerando todos os componentes ideais pode-se fazer a análise do circuito com o objetivo de se obter a relação entre a tensão da bateria e a tensão do painel a partir do fato de que a tensão média no indutor é nula.

Na primeira etapa, com a chave fechada, temos o circuito da Figura \ref{circuito_primeira_etapa}. A tensão no indutor nesta etapa é dada na equação \ref{tensao_indutor_primeira_etapa}

\begin{figure}[!htpb]
\begin{center}
\begin{circuitikz} [american]
\draw
(0,0) to[pvsource, l = V\textsubscript{pv}] (0,3)
(0,3) to[short] (2,3)
(2,3) to[capacitor, l = C\textsubscript{in}] (2,0) -- (0,0)
(2,3) to[inductor, l = L] (4,3)
(4,3) to[short] (4,0) -- (2,0)
(4,3) to[open] (6,3)
(6,3) to[capacitor, l = C\textsubscript{out}] (6,0) -- (4,0)
(6,3) to[short] (8,3)
(8,3) to[battery, v = V\textsubscript{bat}] (8,0) -- (6,0);

\draw[->] (-0.25, 2) -- (-0.25, 2.75) node[midway, left] {I\textsubscript{pv}};
\end{circuitikz}
\end{center}
\caption{Circuito da Primeira Etapa}
\label{circuito_primeira_etapa}
\end{figure}

\begin{equation} \label{tensao_indutor_primeira_etapa}
V_{L} = V_{pv}
\end{equation}

Na segunda etapa a chave é aberta, com isto a corrente no indutor é reduzida, causando uma inversão na polaridade da tensão no indutor, de forma que o diodo é polarizado e passa a conduzir. O circuito desta etapa está na Figura \ref{circuito_segunda_etapa}. A tensão no indutor é dada pela equação \ref{tensao_indutor_segunda_etapa}.

\begin{figure}[!htpb]
\begin{center}
\begin{circuitikz} [american]
\draw
(0,0) to[pvsource, l = V\textsubscript{pv}] (0,3)
(0,3) to[short] (2,3)
(2,3) to[capacitor, l = C\textsubscript{in}] (2,0) -- (0,0)
(2,3) to[inductor, l = L] (4,3)
(4,3) to[capacitor, l = C \textsubscript{out}] (4,0) -- (2,0)
(4,3) to[short] (6,3)
(6,3) to[battery, v = V\textsubscript{bat}] (6,0) -- (4,0);

\draw[->] (-0.25, 2) -- (-0.25, 2.75) node[midway, left] {I\textsubscript{pv}};
\end{circuitikz}
\end{center}
\caption{Circuito da Segunda Etapa}
\label{circuito_segunda_etapa}
\end{figure}

\begin{equation} \label{tensao_indutor_segunda_etapa}
V_{L} = V_{pv} - V_{bat}
\end{equation}

Considerando que a chave fica fechada pelo tempo DT e aberta pelo tempo (1-D)T, onde D é a razão cíclica e T o período, a tensão média no indutor é dada pela equação \ref{tensao_media_indutor}.

\begin{equation} \label{tensao_media_indutor}
\begin{aligned}
V_{L} &= \frac{1}{T}\left(\int_{0}^{DT} V_{pv}\,dt + \int_{DT}^{T} V_{pv} - V_{bat}\,dt\right) \\
&= \frac{1}{T}\left[V_{pv}\cdot DT + (V_{pv}-V_{bat})(T-DT)\right] \\
&= V_{pv}\cdot D + V_{pv} - V_{pv}\cdot D - V_{bat} +V_{bat}\cdot D \\
&= V_{pv} + V_{bat}(D-1)
\end{aligned}
\end{equation}

Como a tensão média no indutor é nula temos:

\begin{equation}
\begin{gathered}
V_{pv} + V_{bat}(D-1) = 0 \\
V_{pv} = V_{bat}(1-D) 
\end{gathered}
\end{equation}

Com isso vemos que é possível controlar a tensão do painel diretamente pelo controle da razão cíclica do conversor.

Esta estratégia de controle direto com conversor \textit{Boost} será a implementada neste trabalho, devido ao fato de que os painéis trabalham em torno de \SI{5}{\volt} e as duas baterias em série apresentam uma tensão de \SI{5,4}{\volt} quando estão descarregadas ao limite inferior.

\section{Conclusões}

Como apresentado, obteve-se uma equação que representa a interface entre os painéis solares e a carga do sistema para dois casos, um com a carga representada por um resistor, na qual a tensão é fixada pelo conversor, e um com a carga representada por um barramento de tensão constante, no qual a tensão do painel é fixada diretamente pelo conversor. Com estas equações em mãos é possível implementar os modelos dos conversores em uma simulação.

Outro aspecto interessante é que, assim como no caso deste trabalho, na eventualidade de se haver um problema com o barramento de tensão constante, por exemplo a bateria ser desconectada do sistema, a carga poderia voltar a ser representada por um resistor, retornando ao primeiro caso. Assim o que foi exposto cobre os dois casos possíveis de operação do sistema.

\chapter{Modelagem das Cargas}

\section{Computador de Bordo}

O computador de bordo tem funções como armazenamento dos dados em memória não-volátil, leitura de sensores e aquisição dos dados dos demais módulos do satélite.

Alguns contribuintes para o consumo estão listados na tabela \ref{consumo_computador_bordo}. Como é possível observar, poucos componentes contribuem para a maior parte do consumo. O cálculo da potência consumida é realizado multiplicando a corrente consumida pela tensão de alimentação, de \SI{3.3}{\volt}, exceto em casos especiais onde a equação é fornecida pelo \textit{datasheet} do componente.

\begin{table}[!htpb]
\centering
\begin{tabular}{c c c c}
\\ \hline
Componente & Quantia & Corrente [\SI{}{\milli\ampere}] & Potência [\SI{}{\milli\watt}] \\ \hline \hline
\glsentryshort{imu} (MPU-9250) & 1 & 3.7 \cite{mpu9250} & 12.21 \\
\glsentryshort{imu} (BMX055) & 1 & 5.7 \cite{bmx055} & 18.81 \\
Gerador de Referência & 1 & 0.026 \cite{ref5030}, \cite{msp430f6659} & 0.0008 \cite{ref5030} \\
Amplificador Operacional & 4 & 0.2 \cite{tlv341} & 2.64 \\
\textit{Watchdog} Externo & 1 & 0.025 \cite{tps3823} & 0.0825 \\
microSD & 1 & 0.25 \cite{microSD} & 0.825 \\
Memória não-volátil & 3 & 0.05 \cite{is25lp128} & 0.495 \\
Microcontrolador & 1 & 8.39 \cite{msp430f6659} & 57.1134 \cite{msp430f6659} \\
Sensor de Corrente & 1 & 0.23 \cite{max9934} & 2.277 \\
Resistor Shunt (\SI{0.05}{\ohm}) & 1 & 19.271 & 0.01857 \\ \hline
Total & - & 19.271 & 94.47 \\ \hline
\end{tabular}
\caption{Consumo do Computador de Bordo}
\label{consumo_computador_bordo}
\end{table}

Aqui foram considerados apenas os consumos constantes, ou seja, o aumento no consumo durante comunicações do processador com outros dispositivos não foi considerado por apresentar um valor muito pequeno quando comparado com o constante (\SI{0.05}{\milli\watt} de média).

\section{Rádios e Amplificadores de Potência}

Os dois rádios do satélite têm como função enviar os dados coletados para a terra, através de telemetria. Durante a transmissão o consumo de potência é muito maior do em outros momentos, devido ao fato de que o \gls{pa} de cada rádio fica desativado. Portanto para modelar o consumo deste sistema será considerado apenas o comportamento dinâmico, de acordo com a tabela \ref{consumo_radios}. Como os dados enviados por cada rádio são diferentes, os períodos de transmissão e o tempo ativo também são diferentes.

\begin{table}[!htpb]
\centering
\begin{tabular}{c c c c}
\\ \hline
Componente & Período/ & Corrente [\SI{}{\ampere}] & Potência [\SI{}{\milli\watt}] \\
& Tempo Ativo [\SI{}{\second}] & & \\ \hline \hline
\glsentryshort{pa} (Transceiver) & 60/2 & 0.6 \cite{rf6886} & 1.98 \\
\glsentryshort{pa} (Beacon) & 10/0.6 & 0.6 \cite{rf6886} & 1.98 \\ \hline
Total & - & - & - \\ \hline
\end{tabular}
\caption{Consumo dos Rádios}
\label{consumo_radios}
\end{table}

\chapter{Simulação do Sistema} \label{secao:simulacao_sistema}

\chapter{Teste na Bancada do Sistem Real} \label{secao:teste_sistema}

Para emular a iluminação solar foram utilizados quatro LEDs de alta potência (\SI{100}{\watt} cada), sendo que a potência fornecida para os leds foi controlada de forma a seguir a curva obtida com o modelo citado na seção \ref{secao:simulacao_sistema}.

A carga do sistema foi dividida em duas partes: o aquecimento das baterias foi emulado com resistores conectados a saída de um conversor DC-DC. Foram usados três resistores de \SI{10}{\ohm} e um de \SI{47}{\ohm} em paralelo para se obter uma potência próxima a calculada. O resto da curva foi emulado através de uma fonte de quatro quadrantes operando com tensão positiva e corrente negativa.

Duas baterias li-ion ICR18650-30A foram usadas para armazenar energia e 3 paineis solares com 4 conjuntos em paralelo de 10 células KXOB22-12X1F em série foram usados para fornecer a energia.

A bancada utilizada pode ser vista na figura \ref{figura_bancada_teste}.

As correntes dos painéis estão nas figuras \ref{figura_teste_corrente_painel1}, \ref{figura_teste_corrente_painel2} e \ref{figura_teste_corrente_painel3}. Ao analisar a soma de todas as correntes vemos que o resultado é próximo a simulação realizada.

\begin{figure}[!htpb]
\begin{center}
\includegraphics[scale=0.5]{figures/testPanel1Current.png}
\caption{Corrente do Painel 1}
\label{figura_teste_corrente_painel1}
\end{center}
\end{figure}

\begin{figure}[!htpb]
\begin{center}
\includegraphics[scale=0.5]{figures/testPanel2Current.png}
\caption{Corrente do Painel 2}
\label{figura_teste_corrente_painel2}
\end{center}
\end{figure}

\begin{figure}[!htpb]
\begin{center}
\includegraphics[scale=0.5]{figures/testPanel3Current.png}
\caption{Corrente do Painel 3}
\label{figura_teste_corrente_painel3}
\end{center}
\end{figure}

Da mesma forma, vemos nas figuras \ref{figura_teste_potencia_painel1}, \ref{figura_teste_potencia_painel2} e \ref{figura_teste_potencia_painel3} que a soma das potências ficou próxima ao simulado.

\begin{figure}[!htpb]
\begin{center}
\includegraphics[scale=0.5]{figures/testPanel1Power.png}
\caption{Potência do Painel 1}
\label{figura_teste_potencia_painel1}
\end{center}
\end{figure}

\begin{figure}[!htpb]
\begin{center}
\includegraphics[scale=0.5]{figures/testPanel2Power.png}
\caption{Potência do Painel 2}
\label{figura_teste_potencia_painel2}
\end{center}
\end{figure}

\begin{figure}[!htpb]
\begin{center}
\includegraphics[scale=0.5]{figures/testPanel3Power.png}
\caption{Potência do Painel 3}
\label{figura_teste_potencia_painel3}
\end{center}
\end{figure}

Nas figuras \ref{figura_teste_tensao_painel1}, \ref{figura_teste_tensao_painel2} e \ref{figura_teste_tensao_painel3} podemos ver o \gls{mppt} em ação realizando o controle da tensão.

\begin{figure}[!htpb]
\begin{center}
\includegraphics[scale=0.5]{figures/testPanel1Voltage.png}
\caption{Tensão do Painel 1}
\label{figura_teste_tensao_painel1}
\end{center}
\end{figure}

\begin{figure}[!htpb]
\begin{center}
\includegraphics[scale=0.5]{figures/testPanel2Voltage.png}
\caption{Tensão do Painel 2}
\label{figura_teste_tensao_painel2}
\end{center}
\end{figure}

\begin{figure}[!htpb]
\begin{center}
\includegraphics[scale=0.5]{figures/testPanel3Voltage.png}
\caption{Tensão do Painel 3}
\label{figura_teste_tensao_painel3}
\end{center}
\end{figure}

Por fim temos as tensões das baterias 1 e 2 (figuras \ref{figura_teste_tensao_bat1} e \ref{figura_teste_tensao_bat2}). Vemos que este resultado teve uma mudança significativa em relação a simulação, sendo que a tensão diminuiu mais no teste real. Isso se deu por causa da ausência de perdas na simulação, principalmente no conversor \textit{Boost} que conecta os painéis solares as baterias. Desta forma a quantidade que entra no sistema durante a região iluminada da curva de irradiância é menor e, consequentemente, a energia disponível durante o período de eclipse é menor.

\begin{figure}[!htpb]
\begin{center}
\includegraphics[scale=0.5]{figures/testBat1Voltage.png}
\caption{Tensão da Bateria 1}
\label{figura_teste_tensao_bat1}
\end{center}
\end{figure}

\begin{figure}[!htpb]
\begin{center}
\includegraphics[scale=0.5]{figures/testBat2Voltage.png}
\caption{Tensão da Bateria 2}
\label{figura_teste_tensao_bat2}
\end{center}
\end{figure}

\chapter{Conclusões} \label{secao:conclusoes}

Em relação a modelagem e simulação realizada, podemos considerar que os painéis estão próximos o suficiente do real, como demonstrado primeiramente no capítulo \ref{secao:painel solar} e depois nos capítulos \ref{secao:simulacao_sistema} e \ref{secao:teste_sistema}. Uma possível melhoria futura seria separar cada painel na simulação, de forma que cada um teria seu controle \gls{mppt} próprio, aproximando assim a simulação da realidade. Isto ainda não foi realizado pelo motivo de que cada \gls{mppt} introduz um \textit{Loop} Algébrico (\textit{Algebraic Loop}) na simulação, e sua solução requer métodos mais robustos e melhor controle da ferramenta.

O modelo das cargas foi utilizado tanto na simulação quanto no teste real. Esta decisão foi tomada baseado no fato de que a simulação ainda não havia sido validada. Portanto não seria possível verificar quais são as falhas na simulação caso o sistema completo fosse utilizado. A validação deste modelo será realizada em trabalhos futuros.

Outro ponto interessante é utilizar um modelo físico do conversor \textit{Boost} no lugar da fonte de tensão controlada para facilitar a simulação de não-idealidades. Apesar de que o modelo representou bem o controle da tensão do painel na simulação, as perdas introduzidas pelo conversor no teste real não apareceram na simulação, o que gerou uma discrepância nos resultados. Esta melhoria também será realizada em trabalhos futuros.

De modo geral, pode-se considerar que os objetivos do trabalho foram alcançados, sendo que o sistema foi validado e agora existem um modelo e uma simulação que podem ser utilizados para se ter uma idéia geral do funcionamento do sistema, assim como podem ser utilizados para projetos de novos sistemas no futuro.

\printbibliography

\end{document}
