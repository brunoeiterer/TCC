\chapter{Conversor Boost}
Para possibilitar o controle da tensão do painel solar e possibilitar a implementação do algoritmo MPPT foi utilizado um conversor boost. O circuito completo pode ser visto na figura \ref{circuito_completo}.

\begin{figure}[!htpb]
\begin{center}
\begin{circuitikz} [american]
\draw
(0,0) to[pvsource, l = V\textsubscript{pv}] (0,3)
      to[inductor, l = L] (2,3)
(2,3) to[nos, l = S] (2,0) -- (0,0)
(2,3) to[diode, l = D] (4,3)
(4,3) to[capacitor, l = C] (4,0) -- (2,0)
(5,2.025) node[nigfete, xscale=1, anchor=G, bodydiode, rotate=90] () {}
(6,2.025) node[nigfete, xscale=-1, anchor=G, bodydiode, rotate=90] () {}
(7,3) to[battery, v = V\textsubscript{bat}] (7,0) -- (4,0);

\draw[->] (-0.25, 2) -- (-0.25, 2.75) node[midway, left] {I\textsubscript{pv}};
\end{circuitikz}
\end{center}
\caption{Circuito completo}
\label{circuito_completo}
\end{figure}