\chapter{Painel Solar}

O funcionamento de células solares é baseado no efeito fotovoltaico, ou seja, a geração de tensão ou corrente elétrica a partir da incidência de luz \cite{jager2014}.

\section{Modelo do Painel Solar}

O modelo para uma célula aqui utilizado é um circuito equivalente composto por uma fonte de corrente I\textsubscript{ph}, um diodo D, uma resistência paralela R\textsubscript{p} e uma resistência série R\textsubscript{s}. Este circuito pode ser encontrado na figura \ref{modelo_celula_solar}, onde I\textsubscript{ph} é a corrente fotogerada do painel, I\textsubscript{D} é a corrente do diodo, I\textsubscript{R\textsubscript{p}} é a corrente na resistência paralela, I é a corrente da célula e V é a tensão da célula.

\begin{figure}[!htpb]
\begin{center}
\begin{circuitikz} [american]
\draw
(0,0) to[I, l = I\textsubscript{ph}] (0,3) -- (2,3)
      to[diode, l = D] (2,0) -- (0,0)
(2,3) to[short] (4,3)
(4,3) to[resistor, l = R\textsubscript{p}] (4,0) -- (2,0)
(4,3) to [short] (4.5,3)
(4.5,3) to[resistor, l = R\textsubscript{s}] (6.5,3)
	  to[short, -o] (7,3)
(4,0) to[short, -o] (7,0)
(7,3) to[open, v=V] (7,0);
\draw[->] (6.25, 3.25) -- (7,3.25) node[midway, above] {I};
\draw[->] (2.25, 2.75) -- (2.25, 2) node[midway, right] {I\textsubscript{D}};
\draw[->] (4.25, 2.75) -- (4.25, 2) node[midway, right] {I\textsubscript{R\textsubscript{p}}};
\end{circuitikz}
\end{center}
\caption{Circuito equivalente da célula solar}
\label{modelo_celula_solar}
\end{figure}

A partir da análise do circuito equivalente, temos a seguinte relação entre a corrente e a tensão da celula:

\begin{equation} \label{eq:relacao_corrente_tensao_celula}
I = I_{ph} - I_{D} - I_{R_{p}}
\end{equation}

Por simplicidade e sem perda de precisão I\textsubscript{ph} pode ser determinada diretamente pela corrente de curto-circuito I\textsubscript{sc} do painel (equação \ref{eq:corrente_fotogerada}), assim pode-se obtê-la diretamente dos \textit{datasheets} fornecidos pelos fabricantes.

\begin{equation} \label{eq:corrente_fotogerada}
I_{ph} = I_{sc}
\end{equation}

A corrente no diodo é dada pela equação \ref{eq:corrente_diodo}, onde I\textsubscript{o} é a corrente de saturação na sombra, q é a carga de um elétron, n é o fator de idealidade, k é a constante de Boltzmann e T\textsubscript{c} é a temperatura da célula.

\begin{equation} \label{eq:corrente_diodo}
I_{D} = I_{o}(e^{-\frac{q(V+I\cdot R_{s})}{nkT_{c}}}-1)
\end{equation}

A corrente no resistor R\textsubscript{p} pode ser obtida através da equação \ref{eq:corrente_Rp}, conhecida com lei de Ohm.

\begin{equation} \label{eq:corrente_Rp}
I_{R_{p}} = \frac{V+I\cdot R_{s}}{R_{p}}
\end{equation}

Combinando as equações apresentadas, obtem-se a equação \ref{eq:relacao_corrente_tensao_celula_com_parametros}, que mostra a relação entre a corrente e a tensão da celula solar a partir dos parâmetros do circuito equivalente.

\begin{equation} \label{eq:relacao_corrente_tensao_celula_com_parametros}
I = I_{ph} - I_{o}(e^{-\frac{q(V+I\cdot R_{s})}{nkT_{c}}}-1) - \frac{V+I\cdot R_{s}}{R_{p}}
\end{equation}